\chapter{Conclusions}

\hl{TOCHECK Again, left largely untouched}

In this thesis, we have presented a new SMT interface called \texttt{z3.rkt},
which lets Racket programmers interact with an SMT solver programmatically. We
have demonstrated through examples the simplicity and usefulness of such an
interaction. The power of \texttt{z3.rkt} comes from the facilities provided
by Racket to build abstractions on top of the SMT-solving capabilities of Z3.
From the user's perspective, the integration is seamless and fully
transparent.

Our implementation is open source and freely available at
\begin{center}
\url{http://www.cse.iitk.ac.in/users/karkare/code/z3.rkt/}
\end{center}

\texttt{z3.rkt}, like all large projects, is a work in progress. What has been
implemented as of the writing of this thesis is a useful subset of Z3
functionality, but there are several gaps still to be filled:

\begin{itemize}
\item Supporting more Z3 constructs, including bit-vectors and external theories
\item Deriving new abstractions guided by practical use cases
\item Possibly integrating with other SMT solvers
\end{itemize}

In the long term, we hope the community will find this system useful and will
contribute to the project to solve large practical problems.
