% thesis.tex
%
% This file is root file for an example thesis written using the
% IIT Bombay LaTeX Style file.
% Created by Amey Karkare (21 June 2007)
%
% It is provided without warranty on an AS IS basis.

%=====================================================================
% Read: http://www.cse.iitb.ac.in/karkare/iitbthesis/
%    FAQ.txt     for frequently asked quetions
%    Changes.txt for changes
%    README      for more information
%=====================================================================

%=====================================================================
% DOCUMENT STYLE
%=====================================================================
% IITB PhD Thesis format default settings are:
%   12pt, one-sided printing on a4 size paper
\documentclass[openright,twoside]{iitkthesis}
% For two-sided printing, with Chapter starting on odd-numbered pages,
% use the following line instead:  
%%\documentclass[openright,twoside]{iitbthesis}

%=====================================================================
% OPTIONAL PACKAGES
%=====================================================================
% To include optional packages, use the \usepackage command.
% For e.g., The package epsfig is used to bring in the Encapsulated
%    PostScript figures into the document.
%    The package times is used to change the fonts to Times Roman;
%=====================================================================

%=====================================================================
%  Single counter for theorems and theorem-like environments:
%=====================================================================
\newtheorem{theorem}{Theorem}[chapter]
\newtheorem{assertion}[theorem]{Assertion}
\newtheorem{claim}[theorem]{Claim}
\newtheorem{conjecture}[theorem]{Conjecture}
\newtheorem{corollary}[theorem]{Corollary}
\newtheorem{definition}[theorem]{Definition}
\newtheorem{example}[theorem]{Example}
\newtheorem{figger}[theorem]{Figure}
\newtheorem{lemma}[theorem]{Lemma}
\newtheorem{prop}[theorem]{Proposition}
\newtheorem{remark}[theorem]{Remark}

\usepackage{minted}
\usemintedstyle{bw}

\usepackage{fontspec}
\setmonofont[Scale=0.85]{DejaVu Sans Mono}

\usepackage[sorting=none, maxbibnames=99, minbibnames=99]{biblatex}
\bibliography{citations}
\usepackage{amsmath}
\usepackage{lipsum}
\usepackage{float}
\usepackage[hidelinks]{hyperref}
\usepackage{microtype}
\usepackage[font=small,labelfont=bf]{caption}
\usepackage{fancyref}
\usepackage{color,soul} % for highlights
\usepackage{enumerate}
\usepackage{tabularx}

\setlength{\headheight}{15pt}

\expandafter\def\csname PY@tok@err\endcsname{}

\floatstyle{ruled}
\newfloat{program}{thp}{lop}
\floatname{program}{Figure}

\numberwithin{program}{chapter}

\BeforeBeginEnvironment{program}{\begin{singlespace}}
\AfterEndEnvironment{program}{\end{singlespace}}

%=====================================================================
% End of Preamble, start of document
%

\begin{document}

%=====================================================================
% Include the prelude for Title page, abstract, table of contents, etc
% You need to modify it to contain your details
% prelude.tex
%   - titlepage
%   - dedication (optional)
%   - approval sheet
%   - course certificate
%   - table of contents, list of tables and list of figures
%   - nomenclature
%   - abstract
%============================================================================


\clearpage\pagenumbering{roman}  % This makes the page numbers Roman (i, ii, etc)



% TITLE PAGE
%   - define \title{} \author{} \date{}
\title{How to \LaTeX\ a Thesis}
\author{Siddharth Agarwal}
\date{June, 2012}

%  - Roll number, required for title page, approval sheet, and
%    certificate of course work 
\rollnum{Y7027429} 

%   - The default degree is ``Doctor of Philosophy''
%     (unless the document style msthesis is specified
%      and then the default degree is ``Master of Science'')
%     Degree can be changed using the command \iitbdegree{}
\iitbdegree{Master \TeX nician}

%   - The default report type is preliminary report.
%      * for a PhD thesis, specify \thesis
\thesis
%      * for a M.Tech./M.Phil./M.Des./M.S. dissertation, specify \dissertation
%\dissertation
%      * for a DIIT/B.Tech./M.Sc.project report, specify \project
%\project
%      * for any other type, use  \reporttype{}
%\reporttype{ReportType}

%   - The default department is ``Unknown Department''
%     The department can be changed using the command \department{}
\department{DEPARTMENT OF COMPUTER SCIENCE \& ENGINEERING}

%    - Set the guide's name
\setguide{Prof Amey Karkare}
\setguidedept{Department of Computer Science and Engineering}

%   - once the above are defined, use \maketitle to generate the titlepage
\maketitle

%--------------------------------------------------------------------%
% CERTIFICATE
%     The first page after the title page.
\makecertificate

%--------------------------------------------------------------------%
% DEDICATION
%   Dedications, if any, must be first page after title page.
\begin{dedication}
To my pet rock, Skippy.
\end{dedication}

%--------------------------------------------------------------------%
% COPYRIGHT PAGE
%   - To include a copyright page use \copyrightpage
% \copyrightpage

%--------------------------------------------------------------------%
% ABSTRACT
\begin{abstract}
  Abstract.
\end{abstract}


%--------------------------------------------------------------------%
% CONTENTS, TABLES, FIGURES
\tableofcontents
\listoftables
\listoffigures

%--------------------------------------------------------------------%
% NOMENCLATURE
\begin{nomenclature}
\begin{description}
\item{\makebox[0.75in][l]{$C_1$}} Constant 1

\item{\makebox[0.75in][l]{$V$}}    Voltage 

\item{\makebox[0.75in][l]{\$}}     US Dollars
\end{description}
\end{nomenclature}

\cleardoublepage\pagenumbering{arabic} % Make the page numbers Arabic (1, 2, etc)


%=====================================================================
% Include the technical part of the report
%% \include{chap_intro}             % Chapter 1: Introduction
%% \include{chap_others}            % Other chapters as required
%% \include{chap_conclusions}       % Finally the summary & conclusions

%=====================================================================
% APPENDIX
%  Appendices, if any, must precede the cited literatures.
%  Appendices shall be numbered in Roman Capitals (e.g. Appendix IV)

%% \appendix
%% \include{appendix_something}          

%=====================================================================
% PUBLICATIONS
%  publications if any may be listed after the literature cited.
%% \include{mypubs}

%=====================================================================
% ACKNOWLEDGMENTS
%   This is the last item in the thesis. It should be signed by
%   author, with date.


\chapter{Introduction: SMT solvers and interfaces}
\label{chap:intro}

Satisfiability Modulo Theories (SMT) solvers let programmers specify constraints
over booleans, integers, pure functions and other types, and either come up
with assignments that satisfy these constraints, or, if possible, a proof that
the constraints aren't satisfiable. Over the last few years, SMT solvers using
DPLL(T)~\cite{dpllt:04} and other frameworks have come into their own and can
solve a wide variety of problems using efficient heuristics. Problems they can
attack range from simple puzzles like Sudoku and n-queens, to planning and
scheduling, program analysis~\cite{Gulwani:08}, whitebox fuzz
testing~\cite{Godefroid:08} and bounded model checking~\cite{Armando:09}.

\section{Using SMT solvers}
\label{sec:usingsmt}

Yet, SMT solvers are only used by a small number of experts. It isn't hard to
see why: the standard way for programs to interact with SMT solvers like
Z3~\cite{z3}, Yices~\cite{yices} and CVC3~\cite{cvc3} is via powerful but
relatively arcane C APIs that require the users to know the particular
solver's internals. For example, \Fref{fig:c-prop} lists a C program that
asks Z3 whether the simple proposition $p \wedge \neg p$ is satisfiable.

\begin{program}
\caption{A C program to ask Z3 whether $p \wedge \neg p$ is satisfiable}
\label{fig:c-prop}
\begin{minted}{c}
Z3_config cfg = Z3_mk_config();
Z3_context ctx = Z3_mk_context(cfg);
Z3_del_config(cfg);
Z3_sort bool_sort = Z3_mk_bool_sort(ctx);

Z3_symbol symbol_p = Z3_mk_int_symbol(ctx, 0);

Z3_ast p = Z3_mk_const(ctx, symbol_p, bool_sort);
Z3_ast not_p = Z3_mk_not(ctx, p);

Z3_ast args[2] = {p, not_p};
Z3_ast conjecture = Z3_mk_and(ctx, 2, args);

Z3_assert_cnstr(ctx, conjecture);

Z3_lbool sat = Z3_check(ctx);

Z3_del_context(ctx);
return sat;
\end{minted}
\end{program}

Simultaneously, most SMT solvers also feature interaction via the standard
input language SMT-LIB~\cite{smtlib2:10}. SMT-LIB is \textit{significantly}
easier to use in isolation. The same program in SMT-LIB would look something
like \Fref{fig:smtlib-prop}.

\begin{program}
\caption{An SMT-LIB program to check whether $p \wedge \neg p$ is satisfiable}
\label{fig:smtlib-prop}
\begin{minted}{scheme}
; Declare a variable we don't know the value of yet
(declare-fun p () Bool)
; Try to find a value satisfying a contradiction
(assert (and p (not p)))
(check-sat)
; Prints "unsat", meaning "unsatisfiable"
\end{minted}
\end{program}

\hl{TOCHECK Simultaneous linear equations are new.}

The complexity of the C interface keeps going up as we move to less trivial
assertions. \Fref{fig:c-simultaneous} is a C program that asks Z3 to solve the
two simultaneous equations $2x\,+\,3y = 5, 4x\,+\,5y = 7$. (The solution is $x =
-2, y = 3$.)

\begin{program}
\caption{A C program to ask Z3 to solve two simultaneous linear equations}
\label{fig:c-simultaneous}
\begin{minted}{c}
Z3_config cfg = Z3_mk_config();
Z3_set_param_value(cfg, "MODEL", "true");
Z3_context ctx = Z3_mk_context(cfg);
Z3_del_config(cfg);

Z3_sort int_sort = Z3_mk_int_sort(ctx);

Z3_symbol symbol_x = Z3_mk_int_symbol(ctx, 0);
Z3_symbol symbol_y = Z3_mk_int_symbol(ctx, 1);

Z3_ast x = Z3_mk_const(ctx, symbol_x, int_sort);
Z3_ast y = Z3_mk_const(ctx, symbol_x, int_sort);
Z3_ast num2 = Z3_mk_int(ctx, 2, int_sort);
Z3_ast num3 = Z3_mk_int(ctx, 3, int_sort);
Z3_ast num4 = Z3_mk_int(ctx, 4, int_sort);
Z3_ast num5 = Z3_mk_int(ctx, 5, int_sort);
Z3_ast num7 = Z3_mk_int(ctx, 7, int_sort);

Z3_ast eq1 = Z3_mk_eq(ctx, Z3_mk_add(ctx, Z3_mk_mul(ctx, num2, x),
                                          Z3_mk_mul(ctx, num3, y)),
                           num5);
Z3_ast eq2 = Z3_mk_eq(ctx, Z3_mk_add(ctx, Z3_mk_mul(ctx, num4, x),
                                          Z3_mk_mul(ctx, num5, y)),
                           num7);

Z3_assert_cnstr(ctx, eq1);
Z3_assert_cnstr(ctx, eq2);

Z3_model m;
Z3_lbool sat = Z3_check_and_get_model(ctx, &m);

// Z3_L_TRUE means satisfied
if (sat == Z3_L_TRUE) {
  Z3_ast xsolved, ysolved;
  // Omitting some error checking
  Z3_eval(ctx, m, x, &xsolved);
  Z3_eval(ctx, m, x, &ysolved);
  int xval, yval;
  Z3_get_numeral_int(ctx, xsolved, &xval);
  Z3_get_numeral_int(ctx, xsolved, &yval);
  printf("x = %d, y = %d", xval, yval);
}
else {
  printf("No solution to equations");
}
\end{minted}
\end{program}

The SMT-LIB interface is still remarkably brief, though (\Fref{fig:smtlib-simultaneous}).

\begin{program}
\caption{SMT-LIB code to solve two simultaneous linear equations}
\label{fig:smtlib-simultaneous}
\begin{minted}{scheme}
(declare-fun x () Int)
(declare-fun y () Int)
(assert (= (+ (* 2 x) (* 3 y)) 5))
(assert (= (+ (* 4 x) (* 5 y)) 7))
(check-sat)
(eval x)
(eval y)
\end{minted}
\end{program}

However, the SMT-LIB interfaces are generally hard to use directly from C
programs and often not as full-featured\footnote{Z3, for instance, supports
plugging in external theories via the C API, but not via the textual SMT-LIB
interface.}  or extensible. Importantly, it is difficult to write programs
that \textit{interact} with the solver in some way, for example by adding
assertions based on generated models. This makes it difficult to build new
abstractions to enhance functionality.

\subsection*{Summary}

In this chapter, we saw what SMT solvers are, how they work, and the power
that they have. We also saw why few programmers use them now, instead
preferring to hand-code cumbersome search and backtracking algorithms.

\chapter{A new interface: \texttt{z3.rkt}}

In \Fref[plain]{chap:intro}, we demonstrated how cumbersome SMT solvers are to
use. Indeed, we faced the same issues while exploring novel methods to verify,
debug and test functional programs. It felt like the C interface was
hamstringing us, and the SMT-LIB interface was good for basic explorations but
not anything more complicated.

We decided to attempt to solve this: our goal was to implement an SMT-LIB-like
interface in a way that allowed for the same power as the C interface while
appearing naturally integrated into a host language. Since SMT-LIB is {\em
s-expression}-based, for the host language a Lisp dialect was a natural
choice. We chose Racket~\cite{racket}, a popular dialect of
Scheme~\cite{scheme-r5rs,scheme-r6rs},
for our implementation, which we call \texttt{z3.rkt}. Racket has
extensive facilities for implementing new languages~\cite{Tobin-Hochstadt:11},
not just for the interface to the solver, but also for the resulting tools
that the solver would make possible.

Using this system, the program to check whether a contradiction is satisfiable
(\Fref{fig:rkt-prop}) becomes almost as brief as the SMT-LIB version. The
program to solve two simultaneous linear equations (\Fref{fig:rkt-simultaneous}) is
similarly brief.

\begin{program}
\caption{Using \texttt{z3.rkt} to determine whether $p \wedge \neg p$ is satisfiable}
\label{fig:rkt-prop}
\begin{minted}{scheme}
(smt:with-context
 (smt:new-context)
 (smt:declare-fun p () Bool)
 (smt:assert (and/s p (not/s p)))
 (smt:check-sat))
\end{minted}
\end{program}

\begin{program}
\caption{Solving simultaneous linear equations with \texttt{z3.rkt}}
\label{fig:rkt-simultaneous}
\begin{minted}{scheme}
(smt:with-context
 (smt:new-context)
 (smt:declare-fun x () Int)
 (smt:declare-fun y () Int)
 (smt:assert (=/s (+/s (*/s 2 x) (*/s 3 y)) 5))
 (smt:assert (=/s (+/s (*/s 4 x) (*/s 5 y)) 7))
 (smt:check-sat)
 (values (smt:eval x) (smt:eval y)))
\end{minted}
\end{program}

\section{Interactive SMT solving: two examples}
\label{sec:interactive}

To demonstrate the value in integrating a language with an SMT solver, we turn
our attention to a pair of classic logical puzzles.

\subsection{Sudoku}

A Sudoku puzzle asks the player to complete a partially pre-filled 9$\times$9
grid with the numbers 1 through 9 such that no row, column, or 3$\times$3 box
has two instances of a number. This is a classic constraint satisfaction
problem, and any constraint solver can handle it with ease.

\Fref{fig:sudoku} lists a Racket program using \texttt{z3.rkt} to solve Sudoku.

\begin{program}
\caption{Racket code using \texttt{z3.rkt} to solve Sudoku}
\label{fig:sudoku}
\begin{minted}{scheme}
(define (solve-sudoku grid)
  (smt:with-context
   (smt:new-context)
   ; Declare a scalar datatype (finite domain type) with 9 entries
   (smt:declare-datatypes ()
     ((Sudoku S1 S2 S3 S4 S5 S6 S7 S8 S9)))
   ; Represent the grid as an array from integers to this type
   (smt:declare-fun sudoku-grid () (Array Int Sudoku))
   ; Assert the standard grid rules (row, column, box)
   (add-sudoku-grid-rules)
   ; Add pre-filled entries
   (add-grid grid)
   (define sat (smt:check-sat))
   ; 'sat means we found a solution, 'unsat means we didn't
   (if (eq? sat 'sat)
       ; Retrieve the values from the model
       (for/list ([x (in-range 0 81)])
         (smt:eval (select/s sudoku-grid x)))
       #f)))
\end{minted}
\end{program}

Here we omit a couple of function definitions: \texttt{add-sudoku-grid-rules}
asserts the standard Sudoku grid rules, and \texttt{add-grid} reads a
partially filled grid in a particular format and creates assertions based on
it. We note that the function \texttt{(select/s arr x)} retrieves the value at
\texttt{x} from the array \texttt{arr}, and that this can be used to add
constraints on the array (for instance, \texttt{(smt:assert (=/s (select/s arr
x) y))}). We also note that if a set of constraints is satisfiable, Z3 can
generate a \textit{model} showing this; values can be extracted out of this
model using the \texttt{smt:eval} command.

However, simply finding a solution isn't enough for a good Sudoku solver: it
must also verify that there aren't any other solutions. The usual way to do
that for a constraint solver is by retrieving a generated model, adding
assertions such that this model cannot be generated again, and then asking the
solver whether the system of assertions is still satisfiable. If it is, a
second solution exists and the puzzle is considered invalid.

In such situations, the interactivity offered by \texttt{z3.rkt} becomes
useful: it lets the programmer add dynamically discovered constraints on the
fly. The last part of the solution might then become something like
\Fref{fig:sudoku-unique}.

\begin{program}
\caption{Ensuring that a Sudoku grid has exactly one solution}
\label{fig:sudoku-unique}
\begin{minted}{scheme}
   ...
   (if (eq? sat 'sat)
       ; Make sure no other solution exists
       (let ([result-grid
         (for/list ([x (in-range 0 81)])
           (smt:eval (select/s sudoku-grid x)))])
         ; Assert that we want a brand new solution by
         ; asserting (not <current solution>)
         (smt:assert
          (not/s (apply and/s
                   (for/list ([(x i) (in-indexed result-grid)])
                     (=/s (select/s sudoku-grid i) x)))))
         (if (eq? (smt:check-sat) 'sat)
             #f ; Multiple solutions
             result-grid))
       #f)))
\end{minted}
\end{program}

This part can even be abstracted out into a function that returns a
lazily-generated sequence of satisfying assignments for any given set of
constraints.

\subsection{Number Mind}

\hl{TOCHECK This section is brand new.}

The deductive game Bulls and Cows, commercialised as
Master~Mind~\cite{mastermind}, is popular all around the world. The rules may
vary slightly, but their essence stays the same: Two players play the game.
One player (we'll call her Alice) thinks of a 4-digit number, and the other
(Bob) tries to find it. Bob guesses a number, and Alice tells him how many
digits he has correct and in the correct position (\textit{bulls}) and how
many he has correct but in the wrong position (\textit{cows}). Through
repeated guessing Bob tries to arrive at the answer.

The game is deceptively simple: while even the standard 4-digit variant is
challenging for humans, the general problem for $n$ digits is
NP-complete~\cite{mastermindnpc}. As such, it becomes an interesting problem for
constraint solvers.

For simplicity, we tackle a variant of the game:
Number~Mind~\cite{numbermind}, where Bob only tells Alice how many digits are
correct and in the correct place (bulls). The user is Alice and the computer
Bob, which means that the game is \textit{interactive}. An API to solve
Number~Mind would have

\begin{enumerate}[(a)]
\item a way to tell the computer how many digits the number has
\item a way for the computer to guess a number
\item a way for the user to tell the computer how many digits it got correct
  in the last guess.
\end{enumerate}

The constraint solver would have an important role in not just (a) and (c) but
also (b), since we would like the computer to make ``reasonable" guesses and
not just wild ones. We do this by never guessing a number that would be
impossible because of the answers already given.

Our system makes all three tasks simple. \Fref{fig:numbermind} defines three
functions, each corresponding to one of the tasks above.

\begin{program}
\caption{Solving Number Mind using \texttt{z3.rkt}}
\label{fig:numbermind}
\begin{minted}{scheme}
; (a) Create variables for each digit
(define (make-variables num-digits)
  (define vars (smt:make-fun/list num-digits () Int))
  ; Every variable is between 0 and 9
  (for ([var vars]) (smt:assert (and/s (>=/s var 0) (<=/s var 9))))
  vars)

; (b) Guess a number. Returns the guess as a list of digits,
; or #f meaning no number can satisfy all the constraints.
(define (get-new-guess vars)
  (define sat (smt:check-sat))
  (if (eq? sat 'sat)
      ; Get a guess from the SMT solver
      (map smt:eval vars)
      #f))

; (c) How many digits the computer got correct. If a digit is
; correct then we assign it the value 1, otherwise 0. We sum up
; the values and assert that that's equal to the number of correct
; digits.
(define (add-guess vars guess correct-digits)
  (define correct-lhs
    (apply +/s
           (for/list ([x guess]
                      [var vars])
             (ite/s (=/s var x)
                    1      ; Correct guess
                    0))))  ; Wrong guess
  (smt:assert (=/s correct-lhs correct-digits)))
\end{minted}
\end{program}

As a demonstration of \texttt{z3.rkt}, we have written a small web application
around the code. The web application is available at

\begin{center}
\url{http://numbermind.less-broken.com}
\end{center}

The source is also available:

\begin{center}
\url{https://github.com/sid0/numbermind}
\end{center}

\section{Design and Implementation}
\label{sec:design-impl}

\texttt{z3.rkt} is currently implemented as a few hundred lines of Racket code
that interface with the Z3 engine via the provided library. Since the system is
still a work in progress, some of these details might change in the future.

\subsubsection{The Z3 wrapper.} We use Racket's foreign interface \cite{racket/foreign}
to map the Z3 library's C functions into Racket. The programmer interface
communicates with Z3 by calling the Racket functions defined by the
wrapper. While it is possible to use the Z3 wrapper directly, we highly
recommend using the programmer interface instead.

\subsubsection{Built-in functions.} Z3 comes with a number of built-in functions that
operate on booleans, numbers, and more complex values. We expose these
functions directly but add a \texttt{/s} suffix to their usual names in the
SMT-LIB standard, because most SMT-LIB names are already defined as functions
by Racket and we want to avoid colliding with them.

\subsubsection{The core commands.} This is a small set of Racket macros and
functions layered on top of the Z3 wrapper. The aim here is to hide the
complexities of the C wrapper (as discussed in \Fref[plain]{sec:usingsmt}) and
stay as close to SMT-LIB version 2 commands \cite{smtlib2:10} as possible. We
prefix commands with \texttt{smt:} to avoid collisions with Racket functions.

\subsection{Derived Abstractions}
\label{sec:derived}

Since the full power of Racket is available to us, we can define abstractions
that allow users to simplify their code. For example, SMT-LIB allows users to
define macros via the \texttt{define-fun} command, as demonstrated by
\Fref{fig:smtlib-max}.

\begin{program}
\caption{An SMT-LIB macro, defined with \texttt{define-fun}}
\label{fig:smtlib-max}
\begin{minted}{scheme}
(define-fun max ((a Int) (b Int)) Int
  (ite (> a b) a b)) ; ite is short for if-then-else
...
(assert (= (max 4 7) 7))
\end{minted}
\end{program}

However, Z3's API exposes no such command. Our first attempt was to define a
Racket function to do the same thing, as in \Fref{fig:rkt-max-fn}.

\begin{program}
\caption{A first attempt at ``macros" in \texttt{z3.rkt}}
\label{fig:rkt-max-fn}
\begin{minted}{scheme}
(define (smt-max a b)
  (ite/s (>/s a b) a b))
...
(smt:assert (=/s (smt-max 4 7) 7))
\end{minted}
\end{program}

This works for smaller macros like \texttt{max}, but in our experience this
sort of na\"{i}ve substitution can result in final expressions for deeply
nested functions becoming too large for Z3 to handle\footnote{In theory, we
could merge common parts of expressions to reduce the number of AST nodes
generated. In our experiments, this proved to be effective, yet still
significantly slower than the solution we finally adopted.}.

We note, however, that any macro can also be written as a universally
quantified formula. For example, \texttt{max} can be rewritten as shown in
\Fref{fig:max-forall}.

\begin{program}
\caption{Macros as universally quantified formulas}
\label{fig:max-forall}
\begin{minted}{scheme}
(declare-fun max (Int Int) Int)
(assert (forall ((a Int) (b Int))
                (= (max a b)
                   (ite (> a b) a b))))
\end{minted}
\end{program}

Indeed, Z3 has a \textit{macro finder} component that identifies and
eliminates universal quantifiers that are macros in disguise. We finally
solved the problem by providing a Racket macro, \texttt{smt:define-fun}, that
has the same syntax as the SMT-LIB command and that performs precisely this
transformation.

\hl{TOCHECK The bit from here until the end of the section is new.}

The definition of \texttt{smt:define-fun} is listed in \Fref{fig:define-fun}.
We use Scheme's \texttt{syntax-rules} macro system \cite[Section~4.3.2]{scheme-r5rs}
to its fullest extent. \texttt{syntax-rules} accepts pairs of input and output
patterns and goes with the output pattern for the first input that can be
matched, somewhat like the \texttt{cond} construct found in many Lisps. We
handle two separate cases: (a) we're defining a plain identifier, in which
case we have no need for the \texttt{forall}, and (b) we're defining a macro
as above, in which case we do. The \texttt{...} as part of the macro
definition is a special form recognized by \texttt{syntax-rules}: wherever it
sees them in the output pattern, it substitutes for them a list of whatever
was present in the input pattern. An example of this substitution is listed in
\Fref{fig:define-fun-example}.

\begin{program}
\caption{A routine for defining macros in \texttt{z3.rkt}}
\label{fig:define-fun}
\begin{minted}{scheme}
(define-syntax smt:define-fun
  (syntax-rules ()
    [(_ id () type body) ; Plain identifiers don't need a forall
     (begin
       (smt:declare-fun id () type)
       (smt:assert (=/s id body)))]
    [(_ id ((argname argtype) ...) type body)
     (begin
       (smt:declare-fun id (argtype ...) type)
       (smt:assert (forall/s ((argname argtype) ...)
                             (=/s (id argname ...) body))))]))
\end{minted}
\end{program}

\begin{program}
\caption{\texttt{smt:define-fun} in action}
\label{fig:define-fun-example}

\begin{minted}{scheme}
(smt:define-fun foo ((x Int) (y Bool)) Int
                    (+/s x (ite/s y 20 0)))
\end{minted}

\begin{center}
{\large $\downarrow$} {\small expands to}
\end{center}

\begin{minted}{scheme}
(smt:declare-fun foo (Int Bool) Int)
(smt:assert (forall/s ((x Int) (y Bool))
                      (=/s (foo x y) (+/s x (ite/s y 20 0)))))
\end{minted}
\end{program}


\subsection{Porting Existing SMT-LIB Code}
\label{sec:porting-smt-lib}

One of our explicit goals is to enable existing SMT-LIB version 2 code to be
ported with a small number of systematic changes. \Fref{tab:smt-porting}
lists the minimal set of changes that needs to be made to port
existing SMT-LIB code to \texttt{z3.rkt}. We expect many SMT-LIB programs
to become shorter as authors use Racket features wherever appropriate.

\begin{table}[hbt]
\caption{Differences between SMT-LIB and \texttt{z3.rkt}}
\label{tab:smt-porting}
\begin{center}
\begin{tabularx}{0.91\textwidth}{lX}
\hline\noalign{\smallskip}
SMT-LIB code & \texttt{z3.rkt} code \\
\noalign{\smallskip}
\hline
\noalign{\smallskip}
Options: \texttt{(set-option :foo true)} & Keyword arguments: \newline \texttt{(smt:new-context \#:foo \#t)} \\

Logics: \texttt{(set-logic QF\_UF)} & The \texttt{\#:logic} keyword: \newline \texttt{(smt:new-context \#:logic "QF\_UF")} \\

Commands: \texttt{declare-fun}, \texttt{assert}, \ldots & Prefixed with \texttt{smt:} \\

Functions: \texttt{and}, \texttt{or}, \texttt{+}, \texttt{distinct} \ldots & Suffixed with \texttt{/s} \\

Boolean literals: \texttt{true} and \texttt{false} & \texttt{\#t} and \texttt{\#f} \\

% Bit-vector literals: \texttt{\#b101}, \texttt{\#x4d56} & As strings: \texttt{"\#b101"}, \texttt{"\#x4d56"} \\
\hline
\end{tabularx}
\end{center}
\end{table}


\printbibliography
\nocite{*}

\end{document}
