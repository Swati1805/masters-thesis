\chapter{Introduction: Satisfiability solvers and interfaces}
\label{chap:intro}

The Boolean satisfiability or SAT problem asks: \textit{Given a boolean
formula with a set of variables in it, is there a way to assign each variable
a value such that the formula becomes true?} The SAT problem is one of the
cornerstones of computer science, with enormous theoretical and practical
implications. Indeed, it was the very first problem to be proved
\textit{NP-complete}~\cite{sat-npcomplete}.

Yet, interest in efficiently solving so-called ``natural'' or ``real-world''
instances of SAT has remained. This is at least partly because a large number
of practical problems are also NP-complete and can be \textit{reduced} to SAT.
Here \textit{reduced} is as it is defined in~\cite{karp21}: roughly, can an
instance of the problem be turned into a boolean formula decided by SAT that's
at most polynomially larger?

\section{Solving SAT: DPLL}

Solving SAT is generally restricted to boolean formulas in conjunctive normal
form (CNF), which consists of \textit{clauses} of boolean
\textit{literals}\footnote{The word \textit{literal} has two different
technical meanings: in computer science, it represents a fixed value such as
the number \texttt{1} or the string \texttt{"Hello"}. In mathematical logic,
it means either an atom (variable: $p$) or its negation ($\neg p$). We use it
here in this latter sense.} joined together with the OR operation ($\vee$),
and these clauses joined together with the AND operation ($\wedge$). It is
proved that the SAT problem for any boolean formula can be reduced to the SAT
problem for formulas in CNF~\cite{karp21}.

CNF formulas have useful properties, such as that they can be represented as a
set of clauses, that at any point a clause once satisfied (shown to be true)
can be dropped from the set, and that showing even one of the clauses to be
false is enough to show that the formula cannot be satisfied.

The simplest way to solve SAT is with a basic backtracking algorithm
(\Fref{fig:sat-backtracking}). The algorithm's recursive nature means it takes
exponential time in the worst case.

\begin{program}

\caption[A na\"{i}ve backtracking algorithm for SAT in Racket]
{A na\"{i}ve backtracking algorithm for SAT in Racket.
\texttt{substitute} substitutes the given value for the atom in the boolean
formula, and returns an updated formula or \texttt{\#f}}

\label{fig:sat-backtracking}
\begin{minted}{scheme}
; atoms is a list of variables to satisfy
(define (solve-sat atoms formula)
  (if (empty? atoms)
      #t   ; no more atoms to substitute means SAT
      (let*
       ([atom (car atoms)]    ; pick the first atom
        [rest-atoms (cdr atoms)]
        [subst-true (substitute formula atom #t)]
        [subst-false (substitute formula atom #f)])

        (or (and subst-true (solve-sat rest-atoms subst-true))
            (and subst-false (solve-sat rest-atoms subst-false))))))
\end{minted}
\end{program}

There are two key insights that can be made here:

\begin{enumerate}

\item \textbf{Unit propagation, or the one-literal clause rule.} \textit{If a
clause contains just one literal, that literal has to be true.} For example,
consider the set of clauses $\{a, a \vee b, \neg a \vee \neg c, b \vee c \vee
d\}$. Since all the clauses need to be true, $a$ must be true. Letting this
happen, we see that $a \vee b$ is true so it is dropped, and $\neg a \vee \neg
c$ simply becomes $\neg c$, which means we are left with the clauses $\{\neg
c, b \vee c \vee d\}$.

At this point unit propagation can be applied once again on $c$, setting it to
false. We are finally left with just the one clause $\{b \vee d\}$.

\item \textbf{Pure literal elimination, or the affirmative-negative rule.}
\textit{If the occurrences of a given variable $p$ are either all $p$ (or all
$\neg p$), then $p$ can be assigned true (or false).} Consider the following
set of clauses: $\{a \vee b, a \vee c, b \vee c \vee \neg d, \neg b \vee \neg
c \vee d\}$. Since $a$ always appears in its positive form, we can assign $a$
true without needing to consider the case where $a$ is false. Thus any clause
that contains $a$ is satisfied and can be dropped. We are left with $\{b \vee
c \vee \neg d, \neg b \vee \neg c \vee d\}$.

\end{enumerate}

The backtracking algorithm, plus these two insights applied at each step, form
the Davis-Putnam-Logemann-Loveland (DPLL) algorithm~\cite{dpll1,dpll2}. The
DPLL algorithm, although over fifty years old, is tremendously successful and
forms the core of many modern complete SAT solvers\footnote{A
\textit{complete} SAT solver is one where a definitive answer is returned,
whether a formula is satisfiable or not. There are also \textit{incomplete}
SAT solvers which return an answer if the formula is satisfiable but do not
return one if it is not. Examples include GSAT and WalkSat~\cite{walksat}.},
including Chaff~\cite{chaff}, GRASP~\cite{grasp} and MiniSat~\cite{minisat}.

Of course, modern SAT solvers employ a large number of heuristic and
optimisation tricks to speed things up. They are highly efficient for
``real-world'' problems and are used in a wide variety of computer science fields,
from automated planning for robots or unmanned vehicles (via the
\textit{satplan} method~\cite{satplan}), to dependency resolution in Linux
package managers such as Zypper~\cite{zypper}. They have also been heavily used in 
program analysis and verification, which is what brings us to the next section.

\section{SMT and DPLL(T): SAT with a twist}

Typically, program analysis tools that used SAT solvers would have to find a
way to translate variables found in programs to boolean ones. For example, a
32-bit integer could be encoded as a set of 32 boolean variables\footnote{It
is also possible to represent integers as boolean variables via predicate
abstraction~\cite{slam}, which is more efficient but potentially loses
information.}.

It was soon realised that pushing this step into the SAT solver would help.
Since users would still be asking whether formulas were satisfiable, except
with variables from more complex domains or \textit{theories}, this approach
was dubbed Satisfiability Modulo Theories (SMT). Early SMT methods
(e.g.~\cite{proc-verification,uclid}) \textit{eagerly} simply converted formulas to
Boolean variables before handing them over to a SAT solver.

Alternatively, one could substitute placeholder boolean variables for theory
variables, supply this \textit{faux} boolean formula to a modified SAT solver,
and write a separate program called a \textit{T-solver} that interacts
\textit{on demand} with the SAT solver to deal with theory variables. The
modified DPLL algorithm is called \textbf{DPLL(T)}~\cite{dpllt:04}, and this
\textit{lazy} approach turns out to be far more efficient than eager methods.
(For a more detailed description of eager and lazy SMT,
see~\cite[Section~3.2]{abstract-dpll}).

State-of-the-art SMT solvers like Z3~\cite{z3}, Yices~\cite{yices} and
CVC3~\cite{cvc3} are based on DPLL(T). They allow users to specify constraints
over booleans, integers, real numbers, arrays, lists, trees, first-order
predicates and other kinds of variables. They either come up with assignments
that satisfy these constraints, or, if possible, a proof that the constraints
aren't satisfiable. SMT solvers have been used to solve problems in planning
and scheduling, program analysis~\cite{Gulwani:08}, whitebox fuzz
testing~\cite{Godefroid:08} and bounded model checking~\cite{Armando:09}.

\section{Using SMT solvers}
\label{sec:usingsmt}

After coming to know of the power SMT solvers have, the first question the
intrepid programmer would ask is how to use one. Unfortunately, the standard
way for programs to interact with SMT solvers is via powerful but relatively
arcane C APIs that require the users to know the particular solver's
internals. For example, \Fref{fig:c-prop} lists a C program that asks Z3
whether the simple proposition $p \wedge \neg p$ is satisfiable.

\begin{program}
\caption{A C program to ask Z3 whether $p \wedge \neg p$ is satisfiable}
\label{fig:c-prop}
\begin{minted}{c}
Z3_config cfg = Z3_mk_config();
Z3_context ctx = Z3_mk_context(cfg);
Z3_del_config(cfg);
Z3_sort bool_sort = Z3_mk_bool_sort(ctx);

Z3_symbol symbol_p = Z3_mk_int_symbol(ctx, 0);

Z3_ast p = Z3_mk_const(ctx, symbol_p, bool_sort);
Z3_ast not_p = Z3_mk_not(ctx, p);

Z3_ast args[2] = {p, not_p};
Z3_ast conjecture = Z3_mk_and(ctx, 2, args);

Z3_assert_cnstr(ctx, conjecture);

Z3_lbool sat = Z3_check(ctx);

Z3_del_context(ctx);
return sat;
\end{minted}
\end{program}

Simultaneously, most SMT solvers also feature interaction via the standard
input language SMT-LIB~\cite{smtlib2:10}. SMT-LIB is \textit{significantly}
easier to use in isolation. The same program in SMT-LIB would look something
like \Fref{fig:smtlib-prop}.

\begin{program}
\caption{An SMT-LIB program to check whether $p \wedge \neg p$ is satisfiable}
\label{fig:smtlib-prop}
\begin{minted}{scheme}
; Declare a variable we don't know the value of yet
(declare-fun p () Bool)
; Try to find a value satisfying a contradiction
(assert (and p (not p)))
(check-sat)
; Prints "unsat", meaning "unsatisfiable"
\end{minted}
\end{program}

The complexity of the C interface keeps going up as we move to less trivial
assertions. \Fref{fig:c-simultaneous} is a C program that asks Z3 to solve the
two simultaneous equations $2x\,+\,3y = 5, 4x\,+\,5y = 7$. (The solution is $x =
-2, y = 3$.)

\begin{program}
\caption{A C program to ask Z3 to solve two simultaneous linear equations}
\label{fig:c-simultaneous}
\begin{minted}{c}
Z3_config cfg = Z3_mk_config();
Z3_set_param_value(cfg, "MODEL", "true");
Z3_context ctx = Z3_mk_context(cfg);
Z3_del_config(cfg);

Z3_sort int_sort = Z3_mk_int_sort(ctx);

Z3_symbol symbol_x = Z3_mk_int_symbol(ctx, 0);
Z3_symbol symbol_y = Z3_mk_int_symbol(ctx, 1);

Z3_ast x = Z3_mk_const(ctx, symbol_x, int_sort);
Z3_ast y = Z3_mk_const(ctx, symbol_x, int_sort);
Z3_ast num2 = Z3_mk_int(ctx, 2, int_sort);
Z3_ast num3 = Z3_mk_int(ctx, 3, int_sort);
Z3_ast num4 = Z3_mk_int(ctx, 4, int_sort);
Z3_ast num5 = Z3_mk_int(ctx, 5, int_sort);
Z3_ast num7 = Z3_mk_int(ctx, 7, int_sort);

Z3_ast eq1 = Z3_mk_eq(ctx, Z3_mk_add(ctx, Z3_mk_mul(ctx, num2, x),
                                          Z3_mk_mul(ctx, num3, y)),
                           num5);
Z3_ast eq2 = Z3_mk_eq(ctx, Z3_mk_add(ctx, Z3_mk_mul(ctx, num4, x),
                                          Z3_mk_mul(ctx, num5, y)),
                           num7);

Z3_assert_cnstr(ctx, eq1);
Z3_assert_cnstr(ctx, eq2);

Z3_model m;
Z3_lbool sat = Z3_check_and_get_model(ctx, &m);

// Z3_L_TRUE means satisfied
if (sat == Z3_L_TRUE) {
  Z3_ast xsolved, ysolved;
  // Omitting some error checking
  Z3_eval(ctx, m, x, &xsolved);
  Z3_eval(ctx, m, x, &ysolved);
  int xval, yval;
  Z3_get_numeral_int(ctx, xsolved, &xval);
  Z3_get_numeral_int(ctx, xsolved, &yval);
  printf("x = %d, y = %d", xval, yval);
}
else {
  printf("No solution to equations");
}
\end{minted}
\end{program}

The SMT-LIB interface is still remarkably brief, though (\Fref{fig:smtlib-simultaneous}).

\begin{program}
\caption{SMT-LIB code to solve two simultaneous linear equations}
\label{fig:smtlib-simultaneous}
\begin{minted}{scheme}
(declare-fun x () Int)
(declare-fun y () Int)
(assert (= (+ (* 2 x) (* 3 y)) 5))
(assert (= (+ (* 4 x) (* 5 y)) 7))
(check-sat)
(eval x)
(eval y)
\end{minted}
\end{program}

However, the SMT-LIB interfaces are generally hard to use directly from C
programs and often not as full-featured\footnote{Z3, for instance, supports
plugging in external theories via the C API, but not via the textual SMT-LIB
interface.}  or extensible. Importantly, it is difficult to write programs
that \textit{interact} with the solver in some way, for example by adding
assertions based on generated models. This makes it difficult to build new
abstractions to enhance functionality.

\subsection*{Summary}

In this chapter, we saw what SAT and SMT solvers are, how they work, and the
power that they have. We also saw why few programmers use SMT solvers now,
instead preferring to hand-code cumbersome search and backtracking algorithms.

\chapter{A new interface: \texttt{z3.rkt}}

In \Fref[plain]{chap:intro}, we demonstrated how cumbersome SMT solvers are to
use. Indeed, we faced the same issues while exploring novel methods to verify,
debug and test functional programs. It felt like the C interface was
hamstringing us, and the SMT-LIB interface was good for basic explorations but
not anything more complicated.

We decided to attempt to solve this: our goal was to implement an SMT-LIB-like
interface in a way that allowed for the same power as the C interface while
appearing naturally integrated into a host language. Since SMT-LIB is {\em
s-expression}-based, for the host language a Lisp dialect was a natural
choice. We chose Microsoft Research's Z3~\cite{z3} as our SMT solver, and
Racket~\cite{racket}, a popular dialect of Scheme~\cite{scheme-r5rs,scheme-r6rs},
for our implementation. We call our implementation \textbf{\texttt{z3.rkt}}. Racket has
extensive facilities for implementing new languages~\cite{Tobin-Hochstadt:11},
not just for the interface to the solver, but also for the resulting tools
that the solver would make possible.

Using this system, the program to check whether a contradiction is satisfiable
(\Fref{fig:rkt-prop}) becomes almost as brief as the SMT-LIB version. The
program to solve two simultaneous linear equations (\Fref{fig:rkt-simultaneous}) is
similarly brief.

\begin{program}
\caption{Using \texttt{z3.rkt} to determine whether $p \wedge \neg p$ is satisfiable}
\label{fig:rkt-prop}
\begin{minted}{scheme}
(smt:with-context
 (smt:new-context)
 (smt:declare-fun p () Bool)
 (smt:assert (and/s p (not/s p)))
 (smt:check-sat))
\end{minted}
\end{program}

\begin{program}
\caption{Solving simultaneous linear equations with \texttt{z3.rkt}}
\label{fig:rkt-simultaneous}
\begin{minted}{scheme}
(smt:with-context
 (smt:new-context)
 (smt:declare-fun x () Int)
 (smt:declare-fun y () Int)
 (smt:assert (=/s (+/s (*/s 2 x) (*/s 3 y)) 5))
 (smt:assert (=/s (+/s (*/s 4 x) (*/s 5 y)) 7))
 (smt:check-sat)
 (values (smt:eval x) (smt:eval y)))
\end{minted}
\end{program}

\section{Interactive SMT solving: two examples}
\label{sec:interactive}

To demonstrate the value in integrating a language with an SMT solver, we turn
our attention to a pair of classic logical puzzles.

\subsection{Sudoku}

A Sudoku puzzle asks the player to complete a partially pre-filled 9$\times$9
grid with the numbers 1 through 9 such that no row, column, or 3$\times$3 box
has two instances of a number. This is a classic constraint satisfaction
problem, and any constraint solver can handle it with ease.

\Fref{fig:sudoku} lists a Racket program using \texttt{z3.rkt} to solve Sudoku.

\begin{program}
\caption{Racket code using \texttt{z3.rkt} to solve Sudoku}
\label{fig:sudoku}
\begin{minted}{scheme}
(define (solve-sudoku grid)
  (smt:with-context
   (smt:new-context)
   ; Declare a scalar datatype (finite domain type) with 9 entries
   (smt:declare-datatypes ()
     ((Sudoku S1 S2 S3 S4 S5 S6 S7 S8 S9)))
   ; Represent the grid as an array from integers to this type
   (smt:declare-fun sudoku-grid () (Array Int Sudoku))
   ; Assert the standard grid rules (row, column, box)
   (add-sudoku-grid-rules)
   ; Add pre-filled entries
   (add-grid grid)
   (define sat (smt:check-sat))
   ; 'sat means we found a solution, 'unsat means we didn't
   (if (eq? sat 'sat)
       ; Retrieve the values from the model
       (for/list ([x (in-range 0 81)])
         (smt:eval (select/s sudoku-grid x)))
       #f)))
\end{minted}
\end{program}

Here we omit a couple of function definitions: \texttt{add-sudoku-grid-rules}
asserts the standard Sudoku grid rules, and \texttt{add-grid} reads a
partially filled grid in a particular format and creates assertions based on
it. We note that the function \texttt{(select/s arr x)} retrieves the value at
\texttt{x} from the array \texttt{arr}, and that this can be used to add
constraints on the array (for instance, \texttt{(smt:assert (=/s (select/s arr
x) y))}). We also note that if a set of constraints is satisfiable, Z3 can
generate a \textit{model} showing this; values can be extracted out of this
model using the \texttt{smt:eval} command.

However, simply finding a solution isn't enough for a good Sudoku solver: it
must also verify that there aren't any other solutions. The usual way to do
that for a constraint solver is by retrieving a generated model, adding
assertions such that this model cannot be generated again, and then asking the
solver whether the system of assertions is still satisfiable. If it is, a
second solution exists and the puzzle is considered invalid.

In such situations, the interactivity offered by \texttt{z3.rkt} becomes
useful: it lets the programmer add dynamically discovered constraints on the
fly. The last part of the solution might then become something like
\Fref{fig:sudoku-unique}.

\begin{program}
\caption{Ensuring that a Sudoku grid has exactly one solution}
\label{fig:sudoku-unique}
\begin{minted}{scheme}
   ...
   (if (eq? sat 'sat)
       ; Make sure no other solution exists
       (let ([result-grid
         (for/list ([x (in-range 0 81)])
           (smt:eval (select/s sudoku-grid x)))])
         ; Assert that we want a brand new solution by
         ; asserting (not <current solution>)
         (smt:assert
          (not/s (apply and/s
                   (for/list ([(x i) (in-indexed result-grid)])
                     (=/s (select/s sudoku-grid i) x)))))
         (if (eq? (smt:check-sat) 'sat)
             #f ; Multiple solutions
             result-grid))
       #f)))
\end{minted}
\end{program}

This part can even be abstracted out into a function that returns a
lazily-generated sequence of satisfying assignments for any given set of
constraints.

\subsection{Number Mind}

The deductive game Bulls and Cows, commercialised as
Master~Mind~\cite{mastermind}, is popular all around the world. The rules may
vary slightly, but their essence stays the same: Two players play the game.
One player (we'll call her Alice) thinks of a 4-digit number, and the other
(Bob) tries to find it. Bob guesses a number, and Alice tells him how many
digits he has correct and in the correct position (\textit{bulls}) and how
many he has correct but in the wrong position (\textit{cows}). Through
repeated guessing Bob tries to arrive at the answer.

The game is deceptively simple: while even the standard 4-digit variant is
challenging for humans, the general problem for $n$ digits is
NP-complete~\cite{mastermindnpc}. As such, it becomes an interesting problem for
constraint solvers.

For simplicity, we tackle a variant of the game:
Number~Mind~\cite{numbermind}, where Bob only tells Alice how many digits are
correct and in the correct place (bulls). The user is Alice and the computer
Bob, which means that the game is \textit{interactive}. An API to solve
Number~Mind would have

\begin{enumerate}[(a)]
\item a way to tell the computer how many digits the number has
\item a way for the computer to guess a number
\item a way for the user to tell the computer how many digits it got correct
  in the last guess.
\end{enumerate}

The constraint solver would have an important role in not just (a) and (c) but
also (b), since we would like the computer to make ``reasonable" guesses and
not just wild ones. We do this by never guessing a number that would be
impossible because of the answers already given.

Our system makes all three tasks simple. \Fref{fig:numbermind} defines three
functions, each corresponding to one of the tasks above.

\begin{program}
\caption{Solving Number Mind using \texttt{z3.rkt}}
\label{fig:numbermind}
\begin{minted}{scheme}
; (a) Create variables for each digit
(define (make-variables num-digits)
  (define vars (smt:make-fun/list num-digits () Int))
  ; Every variable is between 0 and 9
  (for ([var vars]) (smt:assert (and/s (>=/s var 0) (<=/s var 9))))
  vars)

; (b) Guess a number. Returns the guess as a list of digits,
; or #f meaning no number can satisfy all the constraints.
(define (get-new-guess vars)
  (define sat (smt:check-sat))
  (if (eq? sat 'sat)
      ; Get a guess from the SMT solver
      (map smt:eval vars)
      #f))

; (c) How many digits the computer got correct. If a digit is
; correct then we assign it the value 1, otherwise 0. We sum up
; the values and assert that that's equal to the number of correct
; digits.
(define (add-guess vars guess correct-digits)
  (define correct-lhs
    (apply +/s
           (for/list ([x guess]
                      [var vars])
             (ite/s (=/s var x)
                    1      ; Correct guess
                    0))))  ; Wrong guess
  (smt:assert (=/s correct-lhs correct-digits)))
\end{minted}
\end{program}

As a demonstration of \texttt{z3.rkt}, we have written a small web application
around the code. The web application is available at

\begin{center}
\url{http://numbermind.less-broken.com}
\end{center}

The source is also available:

\begin{center}
\url{https://github.com/sid0/numbermind}
\end{center}
