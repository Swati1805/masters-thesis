% prelude.tex
%   - titlepage
%   - dedication (optional)
%   - approval sheet
%   - course certificate
%   - table of contents, list of tables and list of figures
%   - nomenclature
%   - abstract
%============================================================================


\clearpage\pagenumbering{roman}  % This makes the page numbers Roman (i, ii, etc)



% TITLE PAGE
%   - define \title{} \author{} \date{}
\title{Functional SMT solving: A new interface\\\textit{for programmers}}
\author{Siddharth Agarwal}
\date{June, 2012}

%  - Roll number, required for title page, approval sheet, and
%    certificate of course work 
\rollnum{Y7027429} 

%   - The default degree is ``Doctor of Philosophy''
%     (unless the document style msthesis is specified
%      and then the default degree is ``Master of Science'')
%     Degree can be changed using the command \iitbdegree{}
\iitbdegree{Master of Technology}

%   - The default report type is preliminary report.
%      * for a PhD thesis, specify \thesis
\thesis
%      * for a M.Tech./M.Phil./M.Des./M.S. dissertation, specify \dissertation
%\dissertation
%      * for a DIIT/B.Tech./M.Sc.project report, specify \project
%\project
%      * for any other type, use  \reporttype{}
%\reporttype{ReportType}

%   - The default department is ``Unknown Department''
%     The department can be changed using the command \department{}
\department{DEPARTMENT OF COMPUTER SCIENCE \& ENGINEERING}

%    - Set the guide's name
\setguide{Prof Amey Karkare}
\setguidedept{Department of Computer Science \& Engineering}

%   - once the above are defined, use \maketitle to generate the titlepage
\maketitle

%--------------------------------------------------------------------%
% CERTIFICATE
%     The first page after the title page.
\makecertificate

%--------------------------------------------------------------------%
% DEDICATION
%   Dedications, if any, must be first page after title page.
\begin{dedication}
To my grandfather
\end{dedication}

%--------------------------------------------------------------------%
% COPYRIGHT PAGE
%   - To include a copyright page use \copyrightpage
% \copyrightpage

%--------------------------------------------------------------------%
% ABSTRACT
\begin{abstract}
Satisfiability Modulo Theories (SMT) solvers are powerful tools that can quickly solve
complex constraints involving booleans, integers, first-order logic
predicates, lists, and other data types. They have a vast number of potential
applications, from constraint solving to program analysis and verification.
However, they are so complex to use that their power is inaccessible to all
but experts in the field.

We present an attempt to make using SMT solvers simpler by integrating the Z3 solver
into a host language, Racket. Our system defines a programmer's interface in
Racket that makes it easy to harness the power of Z3 to discover solutions to
logical constraints. The interface, although in Racket, retains the
structure and brevity of the SMT-LIB format. We demonstrate this using a range
of examples, from simple constraint solving to verifying recursive functions, all in
a few lines of code. We also discuss other opportunities where our system might find
use.
\end{abstract}


%--------------------------------------------------------------------%
% CONTENTS, TABLES, FIGURES
\tableofcontents
\listoftables
\listoffigures

%--------------------------------------------------------------------%
% NOMENCLATURE
\begin{nomenclature}
\begin{description}
\item{\makebox[0.75in][l]{$C_1$}} Constant 1

\item{\makebox[0.75in][l]{$V$}}    Voltage 

\item{\makebox[0.75in][l]{\$}}     US Dollars
\end{description}
\end{nomenclature}

\cleardoublepage\pagenumbering{arabic} % Make the page numbers Arabic (1, 2, etc)
